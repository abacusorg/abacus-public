\documentclass[11pt,preprint]{aastex}
\usepackage{epsf}

\newcommand{\Peff}{\ensuremath{P_{\rm eff}}}

\newcommand{\mpc}{\ensuremath{{\rm\,Mpc}}}
\newcommand{\impc}{\ensuremath{{\rm\,Mpc}^{-1}}}
\newcommand{\hmpc}{\ensuremath{h^{-1}{\rm\,Mpc}}}
\newcommand{\ihmpc}{\ensuremath{h{\rm\,Mpc}^{-1}}}
\newcommand{\hmpcC}{\ensuremath{h^{-3}{\rm\,Mpc^3}}}
\newcommand{\hgpcC}{\ensuremath{h^{-3}{\rm\,Gpc^3}}}
\newcommand{\ihmpcC}{\ensuremath{h^3 {\rm\,Mpc}^{-3}}}
\newcommand{\kmsmpc}{\ensuremath{{\rm\ km\ s^{-1}\ Mpc^{-1}}}}
\newcommand{\kms}{\ensuremath{\rm\ km\ s^{-1}}}
\newcommand{\msun}{\ensuremath{\rm M_\odot}}

\newcommand{\beq}{\begin{equation}}
\newcommand{\eeq}{\end{equation}}
\newcommand{\beqa}{\begin{eqnarray}}
\newcommand{\eeqa}{\end{eqnarray}}

\newcommand{\bfr}{\ensuremath{{\bf r}}}
\newcommand{\bfv}{\ensuremath{{\bf v}}}
\newcommand{\bfg}{\ensuremath{{\bf g}}}
\newcommand{\bfx}{\ensuremath{{\bf x}}}
\newcommand{\bfq}{\ensuremath{{\bf q}}}
\newcommand{\bfS}{\ensuremath{{\bf S}}}
\newcommand{\bfT}{\ensuremath{{\bf T}}}

\begin{document}

\title{Cosmological Methods in ABACUS}
\author{Daniel Eisenstein, \today}

\section{Cosmological Evolution Equations}

The purpose of this note is to document the differential equations
for evolving the homogeneous cosmological parameters.

The cosmological symplectic leapfrog equations are written in proper
time (Quinn et al. 1997), so it is convenient to pick proper time as
the independent variable.  This will allow us to compute a fine grid
to support the microsteps.

We care about the scale factor $a(t)$ and the Kick and Drift operators.
In units of comoving positions and canonical velocites, the kicks 
occur by the quantity 
\begin{equation}
\Delta \eta_K = \int_{t_1}^{t_2} {dt\over a}
\end{equation}
and the drifts by
\begin{equation}
\Delta \eta_D = \int_{t_1}^{t_2} {dt\over a^2}.
\end{equation}
I choose the notation $\eta$ because $\eta_K$ is the conformal time.

It is convenient, however, to re-scale these functions so that the
Einstein-de Sitter solution is constant.  This will avoid a buildup
of numerical error at early times and avoids the singularity in
$\eta_D$ as $a\rightarrow0$.  Defining
\begin{equation}
f_n(a) = a^n H(a) \int {dt\over a^n} = a^n H(a) \int {da\over a^{n+1} H(a)},
\end{equation}
we have $f_1 = aH\eta_K$ and $f_2 = a^2 H\eta_D$.  In EdS, the solutions
are a constant $2/(3-2n)$.  Note that $f_2$ is negative: $\eta_D = -2a^{-1/2}$.
This is ok: $\eta_D$ is singular at $a=0$, but we only care about differences
in $\eta_D$, which are positive for increasing time.

For $f_n$, we have the differential equation 
\begin{equation}
{df_n\over dt} = H(a)\left[1+nf_n + {1\over 2}f_n {d\ln H^2\over d\ln a}\right].
\end{equation}
We will write an ODE for $a(t)$ below.  As usual, $H(a)$ has a 
simple form
\begin{equation}
H(a) = H_0 \sqrt{ \Omega_m a^{-3} + \Omega_K a^{-2} + \hat\Omega_X }
\end{equation}
where $\hat\Omega_X = \rho_X(a)/\rho_{cr,0}$ is the density of dark
energy at epoch $a$ scaled to the critical density today.  We have
an evolution equation
\begin{equation}
{d\ln\hat\Omega_X\over d\ln a} = -3(1+w)
\end{equation}
for an equation of state $w(a)$.  
The derivative of the Hubble parameter has a simple form:
\begin{equation}
{d \ln H^2\over d\ln a} = { -3a^{-3}\Omega_m -2 a^{-2} \Omega_K -3(1+w) \hat\Omega_X
\over a^{-3}\Omega_m + a^{-2} \Omega_K + \hat\Omega_X}
\end{equation}

For a general choice of $w(a)$, one has to do the differential equation to
find $\hat\Omega_X$ as a function of $a$.
For some choices of $w(a)$, we
can do the integral to get an analytic form for $\hat\Omega_X$.
In particular, the $w_0-w_a$ parameterization $w = w_0+w_a(1-a)$ has the 
solution
\begin{equation}
\hat\Omega_X = \Omega_X a^{-3(1+w_0+w_a)} \exp^{3w_a(a-1)}.
\end{equation}

\bigskip

To integrate $a(t)$, we follow a similar plan to define a scaled function
so that the EdS solution is constant.  We define 
\begin{equation}
\alpha = {2\over 3t} {a^{3/2}\over H_0\sqrt{\Omega_m}},
\end{equation}
which evolves as
\begin{equation}
{d\alpha\over dt} = H\alpha \left({3\over2} - {1\over Ht}\right).
\end{equation}
We have $\alpha=1$ in EdS.  Once solved, we have
\begin{equation}
a = \left({3\alpha t\over 2} H_0\sqrt{\Omega_m}\right)^{2/3}.
\end{equation}

\bigskip

For the growth function $D(a)$, we again define a scaled function $\gamma = D/a$
so that we have a constant in the EdS limit.  However, because the linear
growth function equation is second-order, we need to track $d\gamma/dt$
as well.  This quantity can be connected to the familiar $f = d\ln D/d\ln a$
as 
\begin{equation}
f = {1\over H\gamma}{d\gamma\over dt} +1.
\end{equation}
The equation for $\gamma$ is 
\begin{equation}
{d^2\gamma\over dt^2} = -4H{d\gamma\over dt} - H^2\gamma
\left[ 3 + {1\over2}{d\ln H^2\over d\ln a} - {3\over 2} {\Omega_m H_0^2 a^{-3}\over H^2}
\right],
\end{equation}
which can quickly be split into a coupled first-order ODE for $\gamma$ and 
$d\gamma/dt$.

We choose $\gamma=1$ for the EdS limit.

\bigskip

The solutions for the EdS limit provide adequate initial conditions, but it 
is worth noting that the initial time must be very early if the spatial curvature
is non-zero.  The code currently uses $z=10^6$ as the initial time, so the 
corrections from $\Lambda$ are negligible.  But the corrections from $\Omega_K$
would be $10^{-6}$.  Similarly, if dark energy evolves sufficiently rapidly in
redshift, one might want to correct the initial conditions.

One either picks even earlier redshifts or includes a first-order correction.
For example, I believe that the first-order corrections for $f_n$ are 
\begin{equation}
f_n = {2\over 3-2n}\left[ 1 + {1\over 5-2n}{\Omega_K a\over \Omega_m}
- {3w\over 3-6w-2n} {a^3\hat\Omega_X\over \Omega_m} \right].
\end{equation}
Using $f_0 = Ht$, we can derive the correction for $\alpha$ as
\begin{equation}
\alpha = 1 + {4\over 5}{\Omega_K a\over \Omega_m}
+ {3-3w\over 3-6w} {a^3\hat\Omega_X\over \Omega_m}.
\end{equation}
These have not been checked enough, nor included in the code.

The expressions arise from inserting $f_n = 2/(3-2n) + g_n$ into
\begin{equation}
{df_n\over d\ln a} = 1+nf_n + {1\over 2}f_n {d\ln H^2\over d\ln a}
\end{equation}
to get 
\begin{equation}
{dg_n\over d\ln a} = (n-{3\over2})g_n + \left({g_n\over 2}+{1\over 3-2n}\right)
{d\ln H^2\over d\ln a}.
\end{equation}
Working to lowest order in $d\ln H^2/d\ln a$ allows us to drop the $g_n$
term in the prefactor.  We can then rearrange to get
\begin{equation}
{dg_n a^{-n+3/2}\over d\ln a} = {a^{-n+3/2}\over 3-2n} {d\ln H^2\over d\ln a}.
\end{equation}
Expanding $\ln H^2$ to lowest order is $\Omega_Ka/\Omega_m+\hat\Omega_X a^3/\Omega_m$,
and we can then do the derivative and integral with respect to $\ln a$ to
get the above expression for $f_n$. 

\section{Power Spectrum normalization}

If we define the inverse FFT as
\begin{equation}
f(x) = \sum_k \exp(-ikx) F(k),
\end{equation}
then we have 
\begin{equation}
\left<f(0)^2\right> = \sum_k \left<\left|F(k)\right|^2\right>.
\end{equation}
Meanwhile, we should have that the variance of the density field
should be 
\begin{equation}
\left<\delta^2\right> = \int {d^3 k\over (2\pi)^3} P(k)
= \sum_k {V_k\over (2\pi)^3} P(k) = \sum {1\over V} P(k)
\end{equation}
where $V_k = (2\pi/L)^3$ and $V=L^3$ for a cube of side $L$.

Hence, the Fourier transform of the density field should be normalized
so that the real and complex part each have a variance of $P/2V$.

The displacement fields are then $q_x = (ik_x/k^2) \delta_k$, etc.

In the code we compute the FFTs by layering $\delta$, $q_x$, $q_y$,
and $q_z$ into two complex arrays, with $A = \delta + iq_x$ and $B
= q_y+iq_z$.  The Fourier space arrays are loaded so that the real
and imaginary parts satisfy Hermitian and anti-Hermitian properties,
respectively, under the $k\rightarrow -k$ symmetry.

To make life simpler, we force all elements with at least one
component at the Nyquist frequency to be zero.  This should not
affect the simulation science, as modes just beyond Nyquist have
already been ignored!  But it avoids the bookkeeping headache of
handling the aliasing constraints at the Nyquist frequency.

\end{document}
