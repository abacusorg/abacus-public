\documentclass[11pt,preprint]{aastex}
\usepackage{epsf}

\newcommand{\Peff}{\ensuremath{P_{\rm eff}}}

\newcommand{\mpc}{\ensuremath{{\rm\,Mpc}}}
\newcommand{\impc}{\ensuremath{{\rm\,Mpc}^{-1}}}
\newcommand{\hmpc}{\ensuremath{h^{-1}{\rm\,Mpc}}}
\newcommand{\ihmpc}{\ensuremath{h{\rm\,Mpc}^{-1}}}
\newcommand{\hmpcC}{\ensuremath{h^{-3}{\rm\,Mpc^3}}}
\newcommand{\hgpcC}{\ensuremath{h^{-3}{\rm\,Gpc^3}}}
\newcommand{\ihmpcC}{\ensuremath{h^3 {\rm\,Mpc}^{-3}}}
\newcommand{\kmsmpc}{\ensuremath{{\rm\ km\ s^{-1}\ Mpc^{-1}}}}
\newcommand{\kms}{\ensuremath{\rm\ km\ s^{-1}}}
\newcommand{\msun}{\ensuremath{\rm M_\odot}}

\newcommand{\beq}{\begin{equation}}
\newcommand{\eeq}{\end{equation}}
\newcommand{\beqa}{\begin{eqnarray}}
\newcommand{\eeqa}{\end{eqnarray}}

\newcommand{\bfr}{\ensuremath{{\bf r}}}
\newcommand{\bfv}{\ensuremath{{\bf v}}}
\newcommand{\bfg}{\ensuremath{{\bf g}}}
\newcommand{\bfx}{\ensuremath{{\bf x}}}
\newcommand{\bfq}{\ensuremath{{\bf q}}}
\newcommand{\bfS}{\ensuremath{{\bf S}}}
\newcommand{\bfT}{\ensuremath{{\bf T}}}

\begin{document}

\title{Lagrangian Perturbation Theory from Direct Force Calculations}
\author{Daniel Eisenstein, \today}

\section{Introduction}

Taking $\bfr$ and $\bfv$ to be the comoving position and velocity,
respectively, the equations of motion are
\beqa
{d\bfr\over dt} &=& \bfv \\
{d\bfv\over dt} + 2H\bfv &=& \bfg \\
\nabla_r \cdot \bfg &=& - 4\pi G \rho_{comoving} a^{-3} \delta
\eeqa
In other words, \bfg\ is $a^{-3}$ times the acceleration that one would
derive from the inverse square law computed with comoving (not proper) 
separations.

We define $\bfr = \bfx + \bfq$ where \bfx\ is the 
initial grid and \bfq\ is the comoving Lagrangian displacement.

In linear theory, $\bfg = (3/2) \Omega_m H^2 \bfq$.  This gives rise
to the equation of motion
\beq
{d^2\bfq\over dt^2} + 2H {d\bfq\over dt} = {3\Omega_m H^2\over 2} \bfq
\eeq
For $\Omega_m = 1$, we have $a \propto t^{2/3}$ and $H = 2/3t$.  This 
gives the growing mode solution $\bfq \propto t^{2/3} \propto a$.

\section{Perturbation Theory}

Let's now write 
\beq
\bfq = \epsilon d_1(t) \bfq_1 + \epsilon^2 d_2(t) \bfq_2 + 
	\epsilon^3 d_3(t) \bfq_3 + O(\epsilon^4)
\eeq
This particle distribution produces a gravitational force at the 
location of the particles that we can write
\beq
\bfg = {3\Omega_m H^2\over 2}\left[\epsilon d_1(t) \bfg_1 + 
\epsilon^2 d_2(t) \bfg_2 + \epsilon^3 d_3(t) \bfg_3 + O(\epsilon^4) \right]
\eeq
From linear theory, we know that $\bfg_1 = \bfq_1$.
At this point, these time dependences are just hypotheses; we'll see that 
they hold in $\Omega_m=1$, but not in other cases.

\subsection{Second order}
In particular, for the case of $\bfq = \epsilon d_1(t) \bfq_1$, we must
have a force series of the form
\beq
\bfg = {3\Omega_m H^2\over 2} \left[ \epsilon d_1(t) \bfq_1
	+ \epsilon^2 d_1^2(t) \bfS(\bfq_1)
	+ O(\epsilon^3)\right]
\eeq
where the \bfS\ function is some complicated mode-coupled thing
that is second order in \bfq.  

Considering \bfq\ to second-order, we must have
\beq\label{eq:S}
\bfg = {3\Omega_m H^2\over 2} \left[ \epsilon d_1(t) \bfq_1
	+ \epsilon^2 d_2(t) \bfq_2
	+ \epsilon^2 d_1^2(t) \bfS(\bfq_1)
	+ O(\epsilon^3)\right]
\eeq
Inserting this into the equation of motion, we have
\beq
(\partial_t^2 + 2H\partial_t) (\epsilon d_1 \bfq_1 + \epsilon^2 d_2 \bfq_2)
= {3\Omega_m H^2\over 2} ( \epsilon d_1 \bfq_1 + \epsilon^2 d_2 \bfq_2
	+ \epsilon^2 d_1^2 \bfS(\bfq_1)
\eeq
where $\partial_t$ indicates a derivative with respect to time.
Separating by orders, we recover the linear growth equation
\beq
(\partial_t^2 + 2H\partial_t - {3\Omega_m H^2\over 2}) d_1 \bfq_1 = 0
\eeq
For $\Omega_m = 1$, this has the solution $d_1\propto t^{2/3}$.
The next order is
\beq
(\partial_t^2 + 2H\partial_t - {3\Omega_m H^2\over 2}) d_2 \bfq_2 = 
	{3\Omega_m H^2\over 2} d_1^2 \bfS(\bfq_1)
\eeq
Hence $\bfq_2 = \bfS(\bfq_1)$, and we have a simple ODE for $d_2(t)$.
For $\Omega_m = 1$, the power-law ansatz works and 
we find $d_2(t) = (3/7) d_1^2(t)$, a familiar result.

How to find $\bfS(\bfq_1)$?  If we write the force from
$\bfr = \bfx+d_1 \bfq_1$ as $\bfg[d_1\bfq_1]$, then quick inspection of
(\ref{eq:S}) says that we must have
\beq
d_2 \bfq_2 = {3\over 7} d_1^2 \bfS(\bfq_1) = {3\over 7} {2\over 3\Omega_m H^2} {1\over 2}
	\left( \bfg[d_1\bfq_1] + \bfg[-d_1\bfq_1] \right)
\eeq 
This sum cancels out the $O(\epsilon^3)$ terms.
Hence, two force calculations with opposing first-order displacements isolates
second-order displacement to third-order accuracy.

\subsection{Third order}

Let's continue to third order.  Using the third order \bfq, we must
have
\beq\label{eq:T}
\bfg = {3\Omega_m H^2\over 2} \left[ \epsilon d_1(t) \bfq_1
	+ \epsilon^2 d_2(t) \bfq_2
	+ \epsilon^3 d_3(t) \bfq_3
	+ \epsilon^2 d_1^2(t) \bfS(\bfq_1)
	+ \epsilon^3 d_1^3(t) \bfT_{111}(\bfq_1)
	+ \epsilon^3 d_1(t) d_2(t) \bfT_{12}(\bfq_1)
	+ O(\epsilon^4)\right]
\eeq
For the case of $\Omega_m=1$, we have $d_2\propto d_1^2$, so these
two third order terms have the same time dependence and we can combine
the two vectors into a single vector $\bfT(\bfq_1)$.  Note that this
mode-coupling does depend on the second-order vector \bfS; we suppress
this dependence because we assume that we are using the \bfS\ that comes
from $\bfq_1$.  

Inserting into the equation of motion, we have the third-order terms
\beq
(\partial_t^2 + 2H\partial_t - {3\Omega_m H^2\over 2}) d_3 \bfq_3 = 
	{3\Omega_m H^2\over 2} d_1^3 \bfT(\bfq_1)
\eeq
As before, we find $\bfq_3 = \bfT(\bfq_1)$ and an ODE for $d_3(t)$.
For $\Omega_m = 1$, we find $d_3 = (1/6) d_1^3(t)$.

To find $\bfT$, we use
\beq
\bfg[d_1 \bfq_1 + d_2 \bfq_2] = {3\Omega_m H^2\over 2}
	\left[ d_1 \bfq_1 + d_2 \bfq_2 + d_1^2 \bfS(\bfq_1) + d_1^3 \bfT(\bfq_1)
	\right]
\eeq
So we have
\beq
d_3 \bfq_3 = {1\over 6} d_1^3 \bfT(\bfq_1) = 
	{1\over 6} \left({2\over 3\Omega_m H^2} \bfg[d_1\bfq_1+ d_2\bfq_2]
		- d_1 \bfq_1 - (10/3) d_2 \bfq_2 \right)
\eeq
This difference has errors at $O(\epsilon^4)$, but it is not worth cancelling
them out because we had these errors in our formula for $d_2 \bfq_2$.

Note that we did use the $\Omega_m=1$ assumption here.  At second
order for $\Omega_m \ne 1$, we would not have $d_2 \propto d_1^2$,
but we could still solve the ODE for $d_2$ and proceed.  But at
third order, we needed to use $d_2\propto d_1^2$ to combine two
terms that would otherwise have different time dependences.

For our applications, $\Omega_m$ at high redshift is very close to 1, so
that we are not making a big error.  Still, we should assess the size of it.
Bouchet et al.\ (1995) assert that the $\Omega$ dependence of $d_2$ is amazingly
weak, $\Omega_m^{1/143}$, relative to the $(3/7) d_1^2$ factor.  If so, 
then we are completely fine at high redshift and only 1\% off even at
$z=0$!  The behavior of $d_3$ is likely similar, but I haven't checked.

\subsection{Velocities}

Because our expansion for \bfq\ has known time dependence, we can easily
compute the comoving velocities from $\bfv = \partial_t \bfr$.  We generally
have 
\beq
\partial_t d_j = d_j {1\over d_j}{d\,d_j\over da}{da\over dt} = d_j H f_j
\eeq
where $f_j$ is the familiar $d\ln d_j/d\ln a$.  For $\Omega_m=1$, we
have $d_1 \propto a$ and so $f_1 = 1$, $f_2 = 2$, $f_3 = 3$.
For a comoving displacement $\bfq = \sum d_j \bfq_j$, we then have the 
comoving velocity $\bfv = \sum f_j H d_j \bfq_j$.

\subsection{Implementation}

One can implement this as follows.

\begin{enumerate}
\item Generate the initial Fourier modes and Fourier transform 
$i{\bf k} \hat\delta_{\bf k}$ to get the linear displacement field
at the desired redshift.  Apply to the positions.

\item Compute the force $\bfg[d_1 \bfq_1]$.  Store in the velocity.
This is like a Kick.

\item Rearrange the positions to generate $\bfr = \bfx-d_1 \bfq_1$.  This 
can be done because we know the initial grid location (e.g., from the particle
id number).  This can be a surrogate Drift.

\item Compute the force $\bfg[-d_1 \bfq_1]$.  Add to the velocity.
This is like a Kick.

\item Take the position (currently holding the displacement $-d_1\bfq_1$)
and the velocity (currently holding $7H^2 d_2 \bfq_2$) and manipulate
to form the second-order position $\bfx + d_1\bfq_1 + d_2 \bfq_2$ and
second-order velocity $H d_1 \bfq_1 + 2H d_2 \bfq_2$.  Store in the 
position and velocity.  This can be a surrogate Drift.

\item Compute the force from the new position $\bfg[d_1\bfq_1+ d_2\bfq_2]$.
Multiply by $2/3H^2$.
Subtract the quantity $d_1 \bfq_1 - (10/3) d_2 \bfq_2$, which is currently
encoded as $(7/3)\bfv/H - (4/3)\bfq$.  
Divide by 6 to get $d_3 \bfq_3$.
Add this to the position and 3 times this to the velocity.

\end{enumerate}

The result is third-order Lagrangian perturbation theory for the cost of 
3 force evaluations and memory requirements equal to the normal simulation
code.

One should be wary, however, that third-order accuracy doesn't
necessarily change the redshift where one can start by very much.
If one wants 1% precision in the initial conditions, then 2LPT
allows one to advance by a factor of 10 in scale factor relative
to Zel'dovich.  3LPT only gets another factor of 2.

What the accuracy actually is, i.e. what the small parameter of the 
perturbation expansion is, is not so clear.  The Lagrangian displacement
in CDM cosmologies is dominated by scales around 50 Mpc.  So one can have
displacements that are large compared to the grid spacing even though the
overdensity is small.  Putting this another way, nothing in the above
formalism knows about the grid spacing.  The breakdown in perturbation
theory presumably occurs as we approach shell crossing, i.e., so that 
$\Delta q \sim \Delta x$.  This also corresponds to overdensities of order
unity.

Eisenstein, Seo, \& White (2006) equation 9 gives a formula for the rms 
pairwise Lagrangian displacement as a function of separation.  The result
for the variance of the parallel displacement is 
\beq
{V(\Delta q)\over r_{12}^2} = \int {k^2\,dk\over 2\pi^2} P(k) f_\parallel(kr_{12})
\eeq
where 
\beq
f_\parallel(x) = {2\over x^2} \left({1\over3} - {\sin x\over x} - {2\cos x\over x^2}
+ {2\sin x\over x^3} \right)
\eeq 
which converges to $1/5-x^2/84$ at small $x$.  I think that we want this
variance to be small on the grid spacing, using the full power spectrum (just to be conservative).  At $z=0$ for $\sigma_8=0.8$, this value is about 1.7 Mpc/h
for $r_{12}=0.5$ Mpc/h, growing to 2.1 Mpc/h for 0.25 Mpc/h and dropping
to 1.05 Mpc/h for 2 Mpc/h.  To be conservative, we could use 2 Mpc/h as the
rms displacement on the grid scale.  We probably want this to be more like
10\% of the grid spacing, which would be $z=40$!

We'll need to test this in practice to see what comes out.

\end{document}
