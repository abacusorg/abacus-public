Group finding in Abacus works in a hierarchical fashion and is fully
integrated with the cell and slab structure of the rest of the code.
We begin by finding Level 0 (L0) groups with a Friends-of-Friends
(FOF) finder with a relatively generous linking length, typically
0.20--0.25 of the interparticle spacing.  The L0 groups are used
for microstepping and therefore must be found in every full timestep
in which microstepping is being used.  Further, we must find all
groups down to those of 2 particles.  We then optionally search the
L0 groups to find Level 1 (L1) halos and Level 2 (L2) subhalos, with the
idea that the L1 groups are to be used for Halo Occupation Modeling
and the L2 subhalos provide some measure of substructure within the
L1 halos.  The L1 and L2 searches happen with either FOF or the
Spherical Overdensity (SO) method.  L1 halos are fully contained
within their L0 particle sets, and similarly for L2 within L1.  This
would have happened trivially in the case of FOF, but presents a
small additional restriction in the case of SO.  Because we only
care about L1 halos above some minimum size, typically 20--100, we
can restrict this step to the L0 parents of at least that size.

To speed up the group finding work, as part of the full timestep
force calculation, we also compute a density within a scale $b_d$,
in which we accumulate the particle count for separation $r<b_d$
weighted by $1-r^2/b_d^2$.  By choosing $b_d$ to be just slightly
larger than the L0 FOF linking length, we assure that this density
is non-zero for all particles that have a L0 neighbor and zero for
nearly all the rest.  This immediately eliminates many particles
from the L0 work, in particular avoiding the singlets that require
$O(N^2)$ distance comparisons to resolve.  In addition, we will use
this FOF-scale density to decide the centers for our SO L1/L2 finder.

The L0 FOF algorithm is implemented in a 3-part sequence.  First,
we perform FOF group finding within each cell, yielding CellGroups,
including the singlet particles.  Second, we search all pairs of
adjacent cells to find links between CellGroups.  Third, we use the
links to aggregate the CellGroups into our L0 groups, which we also
call GlobalGroups.  The particles in a cell are permuted to be
contiguous within CellGroups, and the GlobalGroups are then simply
lists of the included CellGroups.

Our FOF finder for a specified set of particles, whether those in a 
cell or those in a L0 or L1 parent halo, is as follows.  For sets smaller
than some threshold, chosen to be 70 based on timings, we do the 
usual simple first-in-first-out queue method.  We choose a particle
to start the group, search all the remaining unassigned particles to
find its neighbors, add those neighbors to the queue, and then repeat
until the queue is empty.  That forms a complete FOF group, and one
then starts the next group with the next unassigned particle.

For larger sets, we use additional steps.  Having chosen the primary
particle, we find the distance to all of the unassigned particles.
Those within the linking length $b$ are added to the queue.  We then
partition the unassigned list to separate those particles within 
$sb$ from those outside.  We further keep the queue separated between
those within $(s-1)b)$ from those larger, doing them first.  The
advantage of this is that all queued particles within $(s-1)b$ will
only require searching the unassigned particles within $sb$, thereby
sharply reducing the number of comparisons that must be done.  Only
when the interior queue particles have been completely searched do
we need to re-partition the full list.  In practice, we keep the 
unassigned particles separated into 3 sublists, using $(s-1)b$ and
$sb$ to divide, so that newly found neighbors can be inserted into 
the correct portion of the pending queue.  As is common, these 
lists are all kept in place, simply swapping end particles to move
the boundaries.

We further speed the algorithm by storing the position and original
particle index in a compact quad of floats, aligned to 16 bytes to
aid in SSE/AVX vectorization.  We store the index as a float, which
is exact for lists of $<$8 million particles.  We multiply the
positions by $10^{12}$ so that the square distances can be computed
as the 4-dimensional distance, with the index component underflowing
single-precision math for the comparison to $b^2$.  This permits
easier vectorization.  We additionally find that computing the
square distance to 4 particles at once has a compact AVX encoding
using the horizontal add instruction.  All comparisons are done in
square distance.  We also keep track of a Cartesian min/max bounding
box for all groups.

Having done all of the neighbor searching using and permuting only
these quads, we then use the indices to access the original particle
data.  For L0 finding, we use them to permute the position, velocities,
accelerations, and auxillary array for particle lists in the cells.
Singlet groups whose particle position lie within $b$ of the cell
boundary are tracked as CellGroups; the interior singlets are ignored
from further work.  All multi-particle groups are tracked as
CellGroups.  Every CellGroup has a unique lookup identification
number, given by the $(i,j,k)$ index of the cell and the enumeration
of the CellGroups within the cell.

The work interior to each set of particles (e.g., each cell) is 
separate from all other sets, so we parallelize over each set.
For the L0 search, we consider one slab (index $i$) at a time and 
parallelize over the pencils (index $j$), giving each thread a
range of $j$ and the full range of $k$.  As the number of CellGroups
to be stored in each cell is not known in advance, we use an adaptable
storage class that keeps the results for each pencil contiguous, as 
these are being generated by a single thread and hence have no
contention in writing.


In the next step, we need to search across all cell boundaries to find
pairs of CellGroups that share a pair of particles within a separation
$b$.  As soon as we find any such pair of particles, the CellGroup pair
is marked as a GroupLink, and we can stop searching the particles.
Of course, only particles within $b$ of the relevant face, edge, or 
corner need be considered.  In practice, we find that the selection of
those particles is sufficiently time-consuming that we only separate
particles by the faces and just incur the extra particle searches for
the edges and corners.  As comparisons are symmetric, we compare 
from each cell to 13 of its neighbors, including all of the neighbors
from the previous slab.  

When preparing a face, we limit the work in several ways.  First,
all CellGroups whose bounding box doesn't fall within $b$ of the 
given face can be ignored.  For the remaining CellGroups, we extract
out the particles that are close to the edge, computing a new Cartesian
bounding box for these.  We then compute the midpoint and corner-to-corner
radius of this box.  By preparing all of the needed faces for a cell
at one time, we get substantial cache reuse in accessing the cell's particles.

The task of comparing two faces is then reduced to the comparison
between a small set of points, using the sum of $b$ and the two
radii.  If a pair of points is close enough, and if both are singlets,
then we have found a link.  If one or both of the points are
multiplets, then we have to go to the actual boundary particle lists
to search for pairs separated by $<b$.

The gathering of the links is independent between all pairs of faces.
In practice, we parallelize by giving each thread a range of $j$ and
use the adaptable pencil-based storage class to accumulate the needed
bookkeeping.  The links we find are stored in a single large list, 
in which each thread reserves short segments to fill, with the 
remaining gaps collapsed at the end.  The links are stored simply
as pairs of CellGroup identifiers and are stored twice, in each of
the symmetric orderings.


For the final step, we must connect the links to find the GlobalGroups.
As we plan that groups can subtend up to $G$ cells, we cannot do
this step until links have been found for all slabs within $G$ of
the present one.  At this point, we sort the GroupLink list by the
first identifier of each link and then build a cell-level indexing
for those slabs, so that we can quickly find all links whose first
index is in a cell.

The traversal of links to define a GlobalGroups then proceeds in a 
manner similar to the first-in-first-out queue of the FOF algorithm.
We select a previously unassigned CellGroup to start the GlobalGroup.
We then find all of the links for which this CellGroup is the first 
index and add the CellGroup in the second index to the search queue,
marking it as assigned.  At this point, we set the first index of 
each link to --1, which will mark that it is to be deleted.
We then proceed down the queue, repeating this, save that when we
find a link, we only add its second index to the queue if it is not
already in the queue.  Note that each pair of linked CellGroups
will be tranversed in both directions in this process, causing both
of the links to be marked for deletion.  When the queue is empty,
the list of CellGroups are marked as a GlobalGroup.  GlobalGroups
can contain only one CellGroup, but those with only 1 particle
are not tracked further.

After GlobalGroups for a slab are found, we gather the particles for
further processing, i.e., microstepping and L1/L2 halo finding.  The
gathering keeps CellGroups contiguous, but places all of the CellGroups
in a GlobalGroup together so that the particles are contiguous.  We also
change coordinate system from the cell-centered positions in the parent
code to coordinates centered on the cell of the first CellGroup.  It
is required that the further processing keep the particles in the
same order, as the scattering of particle information back to the
cells is simple.  For example, we set bits in the auxillary field 
to mark all L0 and L1 particles.


For L1/L2 finding, we use either FOF or SO.  For the FOF option,
we simply reuse the base FOF code, operating on one particle set
per L0 group.  This takes longer than finding the original CellGroups
because the GlobalGroups can contain many more particles.  However,
the spherical partitioning completes in an acceptable amount of
time.

For the SO option, we employ a greedy algorithm within each parent
group.  We use the position/index quads of the FOF code.  We select
the particle with the largest FOF-scale density and mark it as the
center.  We then compute the square distance to all other particles,
storing one copy of that list and sort a second copy in increasing
order.  Working from the end, we find the maximum square distance
at which mass interior (which is known simply from the index of
this array) exceeds the required overdensity threshold.  We then
partition the particle and distance arrays based on the square
distance threshold.  These particles are assigned to this group.
We then repeat this whole procedure on the remaining particles,
selecting the densest particle remaining as the new center.  We stop
when there is no particle with a density exceeding some minimum
value.  This minimum density criteria is not applied to the first
center, so there is always at least one SO halo within the parent,
although it may yet be discarded if it does not meet the minimum
mass requirement.

Unlike FOF, the SO method has numerous subsidiary assumptions.  For
example, we begin our search from the particle with the highest
FOF-scale density, which may not be the center that maximizes the 
mass.  Nor do we consider any competition between centers for the
membership of a particle: it is possible that a particle would have
had a higher overdensity if it belonged to another halo that had
a smaller maximum density at its center.  More competitive choices
might produce memberships more similar to tidal radii.  And we stop
our searching at a choice of a threshold density for the central
particle.  This is important for avoiding $O(N^2)$ distance
calculations between all of the remaining lower-density particles 
in the parent group, but if chosen poorly it could cause us to 
miss some viable SO halos.  We pick the threshold central density
to be equal to the overdensity threshold, for reasons to be justified
next.

As mentioned above, our FOF-scale density uses the weighting kernel
$1-r^2/b^2$.  This was done for several reasons.  By avoiding a
discontinuity at $r=b$, it reduces the noise from small perturbations
in the particle positions.  It also greatly reduces the likelihood
that two particles will have the same density, thereby making the
selection of the densest particle unambiguous in nearly all practical
cases.  And it gives a smaller-scale kernel, better to find
substructure and the densest core of a halo.  The volume of the
kernel is $8\pi b^3/15$, for an effective radius of $1.17b$.  If
we are using $b=0.2$ of the interparticle spacing, then a local
smooth density of 180 corresponds to a weighted count of 2.3, coming
from an average of 6 particles.

For a large halo, the local density at the boundary will be
considerably less than the overdensity interior to the boundary.
For example, for the $1/r^2$ density profile of the singular
isothermal sphere, the boundary density is 3 times lower.  Hence,
in large halos, we expect that relatively few particles will
fluctuate from the mean of 2 particles up to the threshold of 6,
successfully reducing the work.  The higher densities of the L2
search are yet more effective in this.

Meanwhile, if we had a halo with 20 particles with overdensity 180,
this would correspond to a radius of about 0.3 times the interparticle
spacing.  If that halo had a true density profile of $1/r^2$, then
we can integrate our kernel around the center to find an expected
weighted count of 9 (for $b=0.2$).  Hence, the halo would have to
be substantially less cuspy to fail to yield a central particle
above our threshold.  It is important in this regard that the 
density kernel use a smaller volume than the volume of the smallest
SO halos we wish to find.

For larger halos, this interior density scales as $M_{180}^{2/3},
making it exceedingly unlikely that a halo will be missed.  Of
course, it may be miscentered and hence misweighed due to variations
in the FOF-scale density, including the chance of lower-mass but
higher-central-density subhalos.

Neither of our methods involves unbinding of fast-moving particles
or other considerations of the phase-space distribution.  Of course,
in dynamically evolving situations, the energy of a single particle
is not conserved and the binding energy is not a guarantee of 
long-term membership in a halo.  


Having found the L1 halos, we compute and report statistics on the
particle distribution.  These include the center of mass position
and velocity, the same for the largest L2 subhalo, the masses of
the 3 largest L2 subhalos, and the L1 mass.  Around both of the
center of masses (L1 and largest L2), we report the 3 eigenvalues
of the velocity dispersion tensor, the radii of various quantiles
of spherical mass, and the radius and value of the maximum circular
velocity, computed in the spherical approximation.  For SO, we 
also report the FOF-scale density and position of the central particle
in the L1 halo.

Further, we output a subset of the particles, with the intention
that they may serve as a location of satellite galaxies in HOD
formulations.  At the beginning of the simulation, we mark some
particles as taggable, setting a bit in their auxillary field.
Typically this is 10\%, set by a hash of the particle id number.
When requested, we output the position, velocity, and particle ID
of the taggable particles in each L1 halo, with the indexing of
these particles stored in the halo statistics file.  We also 
can output all of the taggable particles in a slab that are not
in any L1 halo; the union of these two sets will give a subsample
of the full simulation matter field, with the same particles used
in each time slice.

To support associations between halos at different time steps, we 
output the particle IDs of all tagged particles in each L1 halo,
again with an indexing stored in the halo statistics.  A taggable 
particle becomes permanently tagged if it ever falls within the largest
L2 subhalo of any halo.  The intention is that one can output
the tagged particle IDs along with the halo statistics at dozens
of redshifts, suitable to create a merger tree, as these particles
will tend to follow the core of the L1 halos and can be associated
between the time slices.


Describe more: Why do we ouput L0 halo and field separately in time
slices?  When do we do the subsample outputs vs only the tagged PIDs?


Discuss speed?  Numbers of CG, GG, SO, etc.

